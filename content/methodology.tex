%! Author = vsharma
%! Date = 25.09.2022
% !TeX spellcheck = en_EN

\chapter{Methodology}

\par AWS enables organizations to dynamically scale their applications and infrastructure.
However, due to the continuously addition of new services
and tools, new security challenges are
introduced.
Based on a report, 61 percent of \gls{smb} believe their
cloud
data is at risk and around 70 percent of organizations IT
leaders are concerned about how secure they are in the
cloud \cite{73}.
Although AWS is responsible for securing the infrastructure, it makes it clear to the users to ensure AWS services are properly configured according to best practices to avoid security vulnerabilities.

\par This research emphasizes identifying the efficiency of the open-source security assessment tool, Prowler.
To do so, different approaches are identified. This chapter explains the used approaches.


\section{Identification of Security vulnerabilities and Exploits}

\par AWS provides more than 100 services, and each of these services is vulnerable to security attacks.
These security vulnerabilities can be classified corresponding to each of the AWS services.
This research work is limited to 5 AWS services namely IAM, EC2, S3, RDS, and Lambda.
Columns 1 and 2 in table \ref{tab:classificationofsecurityvulnerabilities} lists the different security
vulnerabilities based on literatures, OWASP vulnerability
list \cite{51}, CISA \cite{52}, IDC \cite{53}.



\section{Prowlers Validation}

\par Prowler scans the AWS account to check for overly permissive IAM permissions, potential vulnerabilities, and best practice violations.
Prowler executes over 40 checks related to HIPAA and GDPR compliance and 49 checks against the CIS AWS Foundations Benchmark.
Prowler constantly checks the AWS accounts against the
CIS benchmarks for the AWS Cloud. Thus, ensuring that the
organization is operating in accordance with guidelines developed by industry consultants, audit and compliance organizations, software developers, operations specialists, security research organizations, government agencies, and legal experts \cite{74}.

\par To determine the efficiency of Prowler against security vulnerabilities two approaches are identified.
The first approach is a test-driven approach, it’s a
manual approach and the second approach uses an open-source application to evaluate the efficiency of Prowler against the identified security vulnerabilities \cite{47}.

\par In the end, another open-source assessment tool called ScoutSuite is used to perform the security assessment of the AWS account.
After the assessment using ScoutSuite finishes, the results are recorded.
The comparison between the assessment performed by the two open-source security
assessment tools are shown in the next chapter.

\subsection{Using Test Driven Approach}

\par A test case is a set of actions performed on a system to determine if it functions correctly. Test cases helps
in determining if different features within a system are performing as expected and to confirm that the system
satisfies all related standards, customer requirements and guidelines \cite{75}. This manual approach determines the
efficiency of Prowler against the identified security vulnerabilities listed in table \ref{tab:classificationofsecurityvulnerabilities}.

\par Prowler provides several checks to perform security assessments.
These checks help determine whether good practices are being followed to secure the set of resources. For each security assessment check in Prowler, we develop test cases to verify the positive and negative scenarios. In the end, the efficiency of Prowler is determined against the identified security vulnerabilities.

\subsubsection{Algorithm and Flowchart}

\begin{figure}
    \centering
    \includegraphics[width=\textwidth]{flowdiagram.pdf}
    \caption{Security vulnerability Assessment flow diagram}
    \label{fig:flowdiagram}
\end{figure}
\begin{itemize}
    \item As an initial step, the process starts by identifying the security vulnerabilities in the five AWS services.
    The different security vulnerabilities are identified
    based on literatures, OWASP vulnerability list
    \cite{51}, CISA \cite{52}, IDC \cite{53}.
    This is highlighted as Step 1 in the flow diagram \ref{fig:flowdiagram}.
\end{itemize}
\begin{itemize}
    \item After the identification of security vulnerabilities, the next step is to identify checks in Prowler that assesses the security vulnerabilities identified in step 1.
    This corresponds to steps 2 and 3 in the flow diagram \ref{fig:flowdiagram}.
\end{itemize}
\begin{itemize}
    \item For each check in Prowler develop the test scenarios in python and verify the test scenarios. The result of the test scenarios is noted. These are steps 4.a and 5 in the flow diagram. This process repeats until all the possible checks are verified successfully.
\end{itemize}
\begin{itemize}
    \item If a check does not exist in Prowler for the security vulnerability, add the security vulnerability in a separate list.
    Step 4.b in the flow diagram \ref{fig:flowdiagram} highlighted this process.
\end{itemize}
\begin{itemize}
    \item In the end, the efficiency of Prowler is calculated between the verified security vulnerabilities using the checks in Prowler and the security vulnerabilities identified initially in Step 1 of the flow diagram.
    This is represented as Step 7 in the flow diagram \ref{fig:flowdiagram}.
\end{itemize}


\par The test cases are developed for each check in Prowler.
As we see from the listing
4.1, the method check15\_require\_upper\_case\_characters() calls the test case methods
for positive and negative scenarios.
The positive scenario is verified using the method
check15\_require\_upper\_case\_characters\_true() and the negative test scenario is verified
using check15\_require\_upper\_case\_characters\_false().

\par Python boto3 \cite{76} library is used in the test
cases.
Boto3 is the Python SDK for AWS
that allows creating, updating, and deleting AWS resources directly from your Python
scripts.

\par As a first step, for each scenario the AWS resource
such as IAM, EC2, etc. are enabled using boto3 \cite{76}.
Upon enabling the resource, the method associated with the resource is determined and used to set the value on the argument, for example, RequireUppercaseCharacters is set to True in this case. After setting the required value on the argument, the check is executed. After the execution of the check finishes, the return code for the test scenario is determined. Once the return code is obtained, it is verified. Only after the verification is successful, the security vulnerability is added to the list of security vulnerabilities verified using prowler.


\lstset{frame=lines}
\lstset{caption={Test code implementation for Prowler assessment}}
\lstset{label={lst:code_check15}}
\lstset{basicstyle=\footnotesize\ttfamily}
\lstset{numberstyle=\tiny\color{cyan}}
\lstset{keywordstyle=\color{red}}
\lstset{identifierstyle=\color{blue}}
\lstset{stringstyle=\color{orange}}
\lstset{commentstyle=\itshape\color{purple!40!black}}
\lstset{
    numbers=left,
    firstnumber=1,
    numberfirstline=true,
    numberstyle=\tiny\color{green},
}

\begin{lstlisting}[language=Python]
	def check15_require_upper_case_characters_true(self):
            client = boto3.client('iam')
            client.update_account_password_policy(RequireUppercaseCharacters=True)
            response = subprocess.run(["./prowler", "-c", "check15"])
            return response.returncode
        def check15_require_upper_case_characters_false(self):
            client = boto3.client('iam')
            client.update_account_password_policy(RequireUppercaseCharacters=False)
            response = subprocess.run(["./prowler", "-c", "check15"])
            return response.returncode
        def check15_require_upper_case_characters(self):
	    uppercasetrue = self.check15_require_upper_case_characters_true()
	    uppercasefalse = self.check15_require_upper_case_characters_false()
	    if uppercasetrue == 0 and uppercasefalse == 3:
	        self.determine_prowler_matrix('RequireUppercaseCharacters')
\end{lstlisting}


\par Similar steps are followed for verifying all the possible scenarios for different Prowler checks. At the end, when all the checks in Prowler are verified, the efficiency of Prowler is calculated.

\subsection{Using Open-Source Application}

\par The term open source refers to something that can be
modified and shared among people.
This is because it is accessible to everyone publicly.
The open-source software application is an application that is distributed with its source code, making it available
for use, modification, enhancement, and distribution with
its original rights \cite{77}.

\par \par In this approach, an open-source web
application developed in Angular.js, and Java Spring boot
are used \cite{78}.
An information system to store data of their employees is
used by every organization regardless it is private or government.
This application is used to organize the tasks and
activities related to \gls{hr} and several other departments.
It is an application-based system that is used by employers to monitor and manage employees from different work locations.
The employee management system provides features that include the ability to add, update, and delete employee information.
The application interface also provides a feature to list all the employees within the organization.
When a new employee joins the company, the information is added to the database.
In case of a mistake in the employee information, the application enables the feature to make corrections to the employee information.
In case an employee leaves the organization, the employee information can be deleted from the application database.

\subsubsection{Infrastructure architecture}

\begin{figure}
    \centering
    \includegraphics[width=\textwidth]{infrastructure_architecture.pdf}
    \caption{Infrastructure architecture}
    \label{fig:infrastructure_architecture}
\end{figure}

\par The architectural view of the open-source
application infrastructure is shown in the figure
\ref{fig:infrastructure_architecture}.
The application employs AWS services namely IAM, RDS, S3, EC2, and Lambda.
Below is a detailed explanation that helps to understand the role of each AWS service.

\par IAM

\par The application requires the creation of buckets, databases, ec2 instances, and lambda function and carries out the
execution of commands to perform security assessment
using Prowler and ScoutSuite \cite{79}.

\par AWS provides standalone policies called managed
policies that are administered and created by AWS. These
policies have their own  \gls{arn} and thus they are called
standalone policies.
The ARN includes the name of the policy for example
arn:aws:iam::aws:policy/AmazonEC2ReadOnlyAccess is an AWS-managed policy \cite{80}.

\par To perform these operations, the user needs to be assigned the required permissions.
AWS IAM service assigns the required permission to the user.
To create the S3 bucket and perform other operation with AWS S3, the user is assigned AmazonS3FullAccess permission \cite{81}.
To launch the ec2 instance and run the application jar, AmazonEC2FullAccess permission is added to the user \cite{82}.
For creating the database, AmazonRDSFullAccess permission is assigned to the user \cite{83}.
To create and execute the lambda function, AWSLambdaFullAccess is added to the user \cite{84}.
\hfill \break



\hfill \break

\par S3

\par The application backend of the employee management application as described above is developed in Spring boot and the frontend is developed in Angular.JS. The developed application is available in version control systems such as GitHub, Bitbucket, etc.
With the help of \gls{ci}/\gls{cd}, the application code is
deployed
in the AWS cloud.
It helps the development team members to continuously
make changes to their code.
This code is than
automatically tested and after the testing is performed
the code gets pushed for building using the associated
build tools such as Maven and finally gets deployed to
the cloud infrastructure environment
\cite{85}.

\par The required application package, Jar in case of spring boot application, and modules in case of angular.js application are stored in S3 buckets.
For the application to be accessed by the HR or employer,
static web hosting is enabled for the bucket containing angular.js application files.


\hfill \break

\par RDS

\par Amazon RDS help to set up, run and scale database services in the cloud.
The employee management application enables adding, updating, and deleting the employee’s information.
This employee information is stored in the database.
The database details are added to the backend spring boot application which connects to the database to perform the required database operation in response to the request from the frontend angular application by making an API call.

\hfill \break
\par EC2

\par The S3 bucket stores the Java spring boot application jar.
The Jar contains the code implemented to perform different application operations.
For the web application to function, the backend application needs to be run.
To do so an EC2 instance is created.
The backend spring boot application jar is downloaded from the S3 bucket to the ec2 instance.
Once the Jar is downloaded, it is run on the EC2 instance
using the command \textit{java -jar applicationname.jar} 
\cite{78}.

\hfill \break
\par Lambda

\par Lambda services execute code without provisioning servers.
The infrastructure contains a lambda function of the
application that makes
an API call to the RDS database.
The function retrieves the information of employees within the organization.
The database details are hardcoded as a connection string within the lambda function.
Upon successful validation of the database details, the request is made to fetch the employee’s information.
\begin{figure}
    \centering
    \includegraphics[width=175mm,height=12cm]{prowlerAssessment.PNG}
    \caption{Prowler security assessment}
    \label{fig:prowlerassessmentreport}
\end{figure}

\par After the AWS cloud infrastructure is configured, it needs to be assessed.
In order to perform the assessment, the user must be provided with additional managed policies namely \textit{SecurityAudit} and \textit{ViewOnlyAccess}.
In addition to providing the permissions to perform the assessment, the user’s permissions to the AWS services much be restricted to read-only permissions after the infrastructure is set up completely.
This ensures complying with the leave privilege principle \cite{80}.

\par To ensure the safety of the infrastructure, a security assessment needs to be performed to identify any security vulnerabilities that might have been introduced while setting up the infrastructure.
This is performed by making use of security assessment tools such as Prowler.
The assessment starts by executing the command \textit{./prowler}.
This assessment report that is created provides details
about the security vulnerabilities associated with different AWS services within the infrastructure.

Figure \ref{fig:prowlerassessmentreport} shows the snippet of the assessment report generated using Prowler.


\hfill \break
\section{ScoutSuite}

\par ScoutSuite is open-source tools.
This tool is used
to perform the assessment of the cloud infrastructure
environment against security vulnerabilities by external
and
internal security analysts.
It is used by external and internal security analysts to assess cloud environments.
To collect configuration data from high security risk areas for manual audit by researchers Scout Suite uses the API exposed by the cloud service provider.
Based on configuration gathered data, it shows security
risks and issues present in the infrastructure \cite{86}
\cite{87}.
\hfill \break
\par Features of ScoutSuite
\begin{itemize}
    \item Persistent monitoring: ScoutSuite is a user-friendly tool that provides the ability to constantly monitor
    the public cloud accounts.
    It helps to review computing resources in order to optimize the workflow by ensuring high performance and ceaseless service availability.
    It ensures tracking of the issues as they occur.
    Persistent monitoring also helps in the evaluation of
    program speed, resource availability, and resource
    levels, to help predict the security attack before it occurs \cite{88}.
\end{itemize}
\begin{itemize}
    \item Agnostic platform: Agnostic Platform refers to tools or applications that are compatible with any cloud
    infrastructure and can be moved without any operational issues to and from different cloud environments.
    An agnostic platform is similar to cloud-native tools which are platform specific and creates visibility issues businesses operating in a hybrid cloud or multi-cloud environment.
    Using cloud-agnostic tools such as ScoutSuite
    benefits from greater visibility across their cloud
    environment and gives them more clarity about how their ecosystem works, leading to better-informed decision making \cite{88}.
\end{itemize}


\hfill \break

\subsection{Deploying ScoutSuite}

\par Scout Suite is a multi-cloud tool namely \gls{gcp}
Google Cloud Platform, AWS, Microsoft Azure, etc.
In this research work, ScoutSuite is deployed on an AWS EC2 instance with an
IAM role configured \cite{87}.

\par Once the EC2 instance is running, the instance is connected.
Connection to the EC2 instance is established using EC2 Instance Connect, Session Manager, SSH client example Putty, and EC2 serial console.
After the EC2 instance is connected, it must be configured.
Configuring the environment can be done by running the
\textit{aws configure} command.
During configuration, it prompts for users access-key,
secret-key, output format and the region.
The user used to configure the environment must
be assigned the
ReadOnlyAccess and SecurityAudit AWS Managed Policies
\cite{87}.
Once configured, the virtual environment must be installed.
Installation of the virtual environment is done by using the command \textit{pip3 install virtualenv}.
After installing virtualenv, a virtual environment must be created, to do this the command \textit{virtualenv -p python3 venv} is executed.
The next step is to install ScoutSuite on the virtual environment.
Installing virtual environment requires running the commands \textit{source venv/bin/activate} and \textit{pip3 install scoutsuite}.
The final step is to test and run ScoutSuite.
Verification of the installation is done by running the command \textit{scout --help}.
The tool's usage syntax for different cloud environments is shown using the help command.
If the user wishes to use ScoutSuite against a specific
AWS IAM role, the command \textit{scout aws --profile my-aws-cli-profile} is executed.
The assessment of the AWS account against the security
vulnerabilities is started by executing the command\textit{scout aws}.
ScoutSuite gathers data from APIs, performs the authenticity of the configured environment variable, and pulls info on the cloud services and various resources.
Once Scout Suite finishes auditing the environment, an
HTML report will be generated \cite{87}.
Unlike checks in Prowler, ScoutSuite has rules.
To verify publicly accessible RDS instances, ScoutSuite uses the rule \textit{rds-instance-publicly-accessible}.
The result is demonstrated in table
\ref{tab:scoutsuiterule} \cite{88}.

\begin{table}[h!]
    \begin{center}
        \caption{ScoutSuite execution result}
        \label{tab:scoutsuiterule}
        \begin{tabular}{|p{1.4cm}|p{1.7cm}|p{5.0cm}|p{6.0cm}|}
            \hline
            \textbf{Result} & \textbf{Severity} & \textbf{Rule} & \textbf{Rule Title}\\
            \hline
            Good & Danger & rds-instance-publicly-accessible & RDS Instance publicly accessible \\
            \hline
        \end{tabular}
    \end{center}
\end{table}


Figure \ref{fig:deployscoutsuite} shows the snippet of
the report generated by using ScoutSuite.

\begin{figure}
    \centering
    \includegraphics[width=\textwidth]{scoutAssessment.PNG}
    \caption{ScoutSuite Report}
    \label{fig:deployscoutsuite}
\end{figure}