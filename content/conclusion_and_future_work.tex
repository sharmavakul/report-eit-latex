%! Author = vsharma
%! Date = 25.09.2022
% !TeX spellcheck = en_EN

\chapter{Conclusion and Future Work}

\par This chapter summarizes the thesis work and contribution, the overall evaluation results, and possible aspects of future work.

\section{Conclusion}

\par This research was aimed to identify and expose the security vulnerabilities of popular AWS Services.
It also aimed at investigating the efficiency of a security framework called Prowler in performing the assessment of security vulnerabilities.


\par A table highlighting the different security vulnerabilities that existed in AWS cloud infrastructure was created.
The AWS service mapping to each security vulnerability
was added in the table.
The AWS cloud environment was assessed using Prowler to
perform the assessment of these security vulnerabilities.

\par A table highlighting the different security vulnerabilities that existed in AWS cloud infrastructure was created.
The AWS service mapping to each security vulnerability was shown in the table.
After the security vulnerabilities were identified, the AWS cloud environment was assessed using Prowler to perform the assessment of these security vulnerabilities.


\par As part of this research work, the efficiency of Prowler was determined using two approaches.
The first approach was a test-driven manual approach, and the second approach used an open-source web application to determine the efficiency of Prowler.
Based on the number of security vulnerabilities assessed using Prowler, the efficiency of Prowler was calculated.
In both approaches, the efficiency of Prowler was
calculated to be greater than 80 \% with respect to the identified security vulnerabilities.


\par Apart from these two approaches, a real-life example of an organization migrating its workload from an on-premises data center to an AWS cloud infrastructure was considered.
During the assessment of the newly set up cloud infrastructure using Prowler, several new security vulnerabilities and misconfigurations assessed using Prowler that were introduced during the migration to the AWS cloud were highlighted.

\par Later, another security assessment tool called ScoutSuite was introduced.
The assessment of the cloud infrastructure was performed using ScoutSuite.
In the end, the assessment performed using Prowler and ScoutSuite was compared.
Based on the comparison result, it was easy to identify that Prowler was able to assess more security vulnerabilities compared to ScoutSuite.
Thus, it can be easily concluded that Prowler is more
efficient than ScoutSuite in performing the security assessment of the AWS cloud against security vulnerabilities.




\section{Future Work}

\par Although this thesis work has explored the efficiency of Prowler in performing the assessment of security vulnerabilities, further studies are needed to confirm the obtained results.
More research should be done along with security experts from the industry in predicting the prominent security vulnerabilities and attacks in the referred five AWS services and other AWS services.
Another topic for additional research would be to better understand the kind of security vulnerabilities and attacks that can lead to compromising the account, especially against cloud providers.

\par A future plan could be to explore the other security issues and vulnerabilities in the cloud computing environment and aim to design a security model using some encryption techniques for data concealment in cloud computing.
Encryption has been a well established mechanism for protecting delicate information and ensure the security, integrity, and confidentiality of the data.
Another area for research could be to question companies using Prowler about their experience, determine
the pros and cons based on their experience, and identify
the limitations that the companies could face while using
Prowler.