%! Author = vsharma
%! Date = 25.09.2022
% !TeX spellcheck = en_EN

\chapter{Conclusion and Future Work}

\par This chapter summarizes the thesis work and contribution, the overall evaluation results, and possible aspects of future work.

\section{Conclusion}

\par This research was aimed to identify and expose the security vulnerabilities of popular AWS Services.
It also aimed at investigating the efficiency of a security framework called Prowler in performing the assessment of security vulnerabilities.


\par Based on literatures, the \gls{owasp} vulnerability list \cite{51}, \gls{cisa} \cite{52}, \gls{idc} \cite{53} different security vulnerabilities in five AWS services were identified.
After identifying the different security vulnerabilities, the AWS cloud environment was assessed using Prowler to perform the assessment of these security vulnerabilities.


\par As part of this research work, the efficiency of Prowler was determined using two approaches.
The first approach was a test-driven manual approach, and the second approach used an open-source web application to determine the efficiency of Prowler.
Based on the number of security vulnerabilities assessed using Prowler, the efficiency of Prowler was calculated.
In both approaches, the efficiency of Prowler was
calculated to be greater than 80 \% with respect to the identified security vulnerabilities.


\par Apart from these two approaches, a use case of
an organization migrating its workload from an on-premises data center to an AWS cloud infrastructure was considered.
During the assessment of the newly set up cloud infrastructure using Prowler, several new security vulnerabilities and misconfigurations assessed using Prowler that were introduced during the migration to the AWS cloud were highlighted.

\par In addition to Prowler, another security assessment
tool called ScoutSuite was introduced.
The cloud infrastructure was assessed against the security
vulnerabilities using ScoutSuite.
In the end, the assessment performed using Prowler and
ScoutSuite was compared.

\par Based on this research work, it can be concluded that Prowler is highly efficient in performing the security assessment of the identified security vulnerabilities.
It provides several checks for assessing different AWS services against security vulnerabilities.
Prowler generates the assessment result in multiple formats such as CSV, HTML, and JSON which enables integration with other tools, for example, Splunk.
It’s auditing, assessment, forensics readiness, and hardening capabilities make the AWS cloud infrastructure highly reliable and less vulnerable to incidents.


\section{Future Work}

\par Although this thesis work has explored the efficiency of Prowler in performing the assessment of security vulnerabilities, further studies are needed to confirm the obtained results.
More research should be done along with security experts from the industry in predicting the prominent security vulnerabilities and attacks in the referred five AWS services and other AWS services.


\par Another area for research could be to question companies using Prowler about their experience, determine
the pros and cons based on their experience, and identify
the limitations that the companies could face while using
Prowler.