%! Author = vsharma
%! Date = 25.09.2022
% !TeX spellcheck = en_EN

\chapter{Conclusion and Future Work}

\par This chapter summarizes the thesis work and contribution, the overall evaluation results, and possible aspects of future work.

\section{Conclusion}

\par This research was aimed to identify and expose the security vulnerabilities of popular AWS services and investigate if existing security frameworks provide enough security to minimize these security vulnerabilities.
A table was created that highlights the security vulnerability and maps the corresponding AWS service.
In this research work, the security vulnerabilities were examined in clouds for only five AWS services (EC2, S3, IAM, RDS, Lambda).
Several tools and security frameworks exist to aid in the scanning of AWS security vulnerabilities and exploit the potential.
The existing frameworks do not protect against the discovered attack surface and are therefore not sufficient to provide a secure cloud environment.
Based on the report by Secura \cite{98}, the CIS Benchmark does not completely secure the AWS environment but provides the required base level of security before implementing other in-depth measures.
To improve the AWS environment security, an improved security framework called Prowler was considered in this research work.
Prowler offers more in-depth security for cloud customers.
Prowler includes the predefined AWS CIS benchmark guidelines and additionally provides checks related to HIPAA, GDPR, PCI-DSS, and others \cite{75}.

\par As part of this research work, the efficiency of Prowler was determined using two approaches.
The first approach was a test-driven manual approach, and the second approach used an open-source web application to determine the efficiency of Prowler.
Based on the security vulnerabilities assessed using the two approaches the efficiency of Prowler was calculated.
Later, another open-source assessment tool called ScoutSuite was introduced.
The result of the assessment performed using Prowler was compared to the assessment performed using ScoutSuite.
In the end, the efficiency of Prowler in assessing the security vulnerabilities was determined to be higher than the assessment performed using ScoutSuite.


\section{Future Work}

\par Although this thesis work has explored the efficiency of Prowler in performing the assessment of security vulnerabilities, further studies are needed to confirm the obtained results.
More research should be done along with security experts from the industry in predicting the prominent security vulnerabilities and attacks in the referred five AWS services and other AWS services.
Another topic for additional research would be to better understand the kind of security vulnerabilities and attacks that can lead to compromising the account, especially against cloud providers.

\par A future plan could be to explore the other security issues and vulnerabilities in the cloud computing environment and aim to design a security model using some encryption techniques for data concealment in cloud computing.
Encryption has been a longstanding way for sensitive information to be protected and ensures the confidentiality, integrity, and security of the data.
Another area for research could be to question the companies using Prowler about their experience, determine the pros and cons based on their experience, and identify the limitations that the companies cloud face while using Prowler.