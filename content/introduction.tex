%! Author = vsharma
%! Date = 25.09.2022
% !TeX spellcheck = en_EN

\chapter{Introduction}

\par Cloud computing is in continual development to make different levels of on-demand services available to customers.
It helps companies reduce operating costs while increasing efficiency.
Organizations are continuously moving their technical assets to cloud, due to the continued proliferation of AWS. Despite several advantages offered by cloud computing, it also fosters security concerns that hamper the fast rate of adoption of cloud computing.
Enterprises should
be aware about the security challenges
faced
by
cloud computing before implementing to cloud
computing by expanding their
on-premises
infrastructure.
Even though cloud computing highly benefits people,
security has been the biggest challenge in cloud.
Comprehending the security threats and countermeasures will help organizations carry out the cost-benefit analysis and impulse the organizations to shift to the cloud.
To provide the cloud users better quality of service, security flaws must
be identified.
Many security vulnerabilities exist in cloud, and hackers
continue exploiting these security holes.
There is a glaring need for a tool that provides a comprehensive and structured approach to pen-testing AWS vulnerabilities by the offensive security community \cite{2}.

\section{Motivation}
\par Cloud computing has become a
popular subject in computer science and software
engineering in recent years.
With low maintenance costs, flexibility, ease of
deployment, cost-effective scalability, and overall aid in efficiency, businesses, and IT leaders all over the world are replacing legacy, classic on-premises data center technology with cloud-based solutions.
As cloud computing offers beneficial characteristics like
on-demand provisioning and billing against the actual
consumption of computing resources, enterprises leverage
these advantages by migrating their costly and legacy
infrastructures such as servers, firewall appliances,
databases, or their software and hardware solutions that
are not operating at the optimum capability to the cloud
such as Microsoft Azure, Amazon Web Services (AWS), Google
Cloud Platform (GCP), or other third-party cloud vendors.
Cloud migration promises cost benefits and agility and it opens the way for the future.
The pandemic fast-tracked the shift to cloud technology by upending how organizations worked and operated.
The lockdown compelled people to work from their homes using conferencing and collaboration services \cite{2}.


\par Cloud computing provides 3 as-a-service offerings
namely PaaS abbreviated as Platform as a Service,
SaaS abbreviated as Software
as a Service
(SaaS), and IaaS abbreviated as Infrastructure
as a
Service
(IaaS).
These cloud
services help enterprises to scale and adopt powerful cloud offerings such as computing and storage capabilities.
The enterprise doesn't need to bear the cost and overhead
of purchasing and maintaining the IT infrastructure in its
data center and can access and scale its IT capabilities at
an anticipated cost.
The term \textit{as a service} refers to the way the IT assets are consumed and shows the major difference between traditional IT and cloud computing.
In traditional IT, a company consumes IT assets such as system software, hardware, applications, and tools.
The company purchases these IT assets and maintains them within the company’s on-premises data center.
In cloud computing, these IT assets are owned, managed,
and maintained by the cloud service providers, the enterprise can use only the infrastructure needed to create the development environment, scale the environment up or down as long as they need, and then stop when finished, paying only for the resources used. This helps enterprises to match the current world demand and reduces expenses \cite{3}.

\par Though cloud migration may seem intimidating at
first, if the preparation is done correctly, it will lead to a smooth migration.
While cloud migration brings greater opportunities, it introduces its own challenges.
These challenges are associated with security controls
that the enterprise built for its on-premises environment.
Security control needs to rebuild, making use of the cloud services to replace traditional on-premises security measures.
When migrating to the cloud for the first time, the required expertise might not be available within the organization.
Another biggest challenge of migrating the application to the cloud is to protect sensitive business data against attacks.
The application migration is oriented mainly toward moving the application into the cloud.
However, legacy applications need a more sophisticated migration mechanism to find out the most suitable cloud environment and fully leverage the benefits of cloud computing \cite{4}.

\par Compliance and security are shared responsibilities between AWS and the customer.
The shared security responsibility model distributes these
responsibilities between the customer and the cloud
service providers.
The security team of the customer maintains the
responsibilities for security as they move applications,
containers, data, and workloads to the cloud, while the
cloud service provider takes some responsibility, but not
all, thus reducing the workload for the customer.
However, this notion of shared responsibility can be misunderstood and can result in the assumption of the cloud
workloads, as well as any data, applications, or activity associated with them are fully protected by the cloud provider \cite{5}.



\par FlexBooker, an online booking software was hacked on
December 23, 2021, \cite{6}\cite{7}.
This hacking resulted in the looting of approximately 3
million
users personal information such
as driver's license data, etc.
FlexBooker hosted its application and services in the AWS cloud, using computing and storage instances.
The data was exposed due to an improperly configured S3 bucket \cite{8}.
Amazon S3 is a storage service that AWS offers, and it
stores objects within buckets.
These objects represent a file and its descriptive metadata \cite{9}.




\par  To protect users from such security attacks, several security frameworks exist that aims to strengthen the
security of users account.
These frameworks provide a list of controls and are recommended with different cloud services and
configurations.
But such frameworks do not always provide in-depth security controls for the services.
Cloud
environments assessed through such frameworks might still have gaps and are vulnerable to attacks.
The reason is simple, these frameworks are not
derived from vulnerability data or actual exploits \cite{10}.

\par This research work aims at identifying the competency of a security framework named \textit{Prowler}.
It runs security checks and performs assessments of different security vulnerabilities such as misconfiguration,
server-side request forgery, etc.
This thesis focuses on different AWS services such as
\gls{ec2}, \gls{rds},
\gls{iam}, Lambda, and \gls{s3}.
The
framework assesses different vulnerabilities that occur on the mentioned AWS services.



\section{Goals}
\par In this regard, the goal of this thesis is to acquaint users with different security vulnerabilities in AWS
services.
These security vulnerabilities are assessed using an assessment tool called \textit{Prowler}. The
efficiency of assessment done using Prowler is determined using 2 approaches. In the first approach, test scenarios
are developed to verify the assessment results for each check in Prowler. For the second approach, an open-source
application is used and an assessment of the security vulnerabilities within the AWS account is done.

\par In the end, the assessment of security vulnerabilities performed using Prowler is compared to the assessment performed by other security assessment tools \textit{ScoutSuite}.

\par For example, if we have an AWS account with a public S3 bucket, it is a security vulnerability and can lead to a data breach.
By using Prowler, it is possible to identify such security vulnerabilities and thus limit the risk of security attacks.

\section{Contribution and Thesis Outline}
\par The thesis contributions are:
\begin{itemize}
    \item Identify the security vulnerabilities in AWS services.
\end{itemize}
\begin{itemize}
    \item Perform the security vulnerability assessment using Prowler.
\end{itemize}
\begin{itemize}
    \item Compare the result of the assessment done using Prowler against ScoutSuite.
\end{itemize}
\begin{itemize}
    \item Determine the efficiency of Prowler against the identified security vulnerabilities.
\end{itemize}

\par This thesis is organized as follows:
\par \textbf{Chapter 2 (Related Work)}: This chapter briefs about:
\begin{itemize}
    \item Cloud computing: The basics of cloud computing,
    its types, and the different types of cloud services
    namely IaaS, PaaS, and SaaS.
\end{itemize}
\begin{itemize}
    \item Amazon Web Services (AWS): A detailed
    background about Amazon Web Services (AWS) and AWS
    services such as RDS, Lambda, S3, etc.
\end{itemize}
\begin{itemize}
    \item Security framework: Information about security
    framework basics and an open-source security assessment tool, Prowler is described.
\end{itemize}

\par \textbf{Chapter 3 (Problem Statement)}: In this chapter, different security vulnerabilities identified in five
AWS services are highlighted.
A piece of detailed information about each of the identified security vulnerabilities is also provided in this chapter.
\par \textbf{Chapter 4 (Methodology)}: This chapter describes the approaches to determine the efficiency of prowler.
\begin{itemize}
    \item Using Test Driven Approach: The manual approach of developing test cases to determine the efficiency of Prowler.
\end{itemize}
\begin{itemize}
    \item Using Open-Source Application: Evaluate the efficiency of Prowler by deploying an open-source
    application on AWS.
\end{itemize}
\par In the end, another open-source tool ScoutSuite is
described, and an AWS account is assessed using ScoutSuite.

\par \textbf{Chapter 5 (Evaluation)}: This chapter provides
\begin{itemize}
    \item The result of the assessment performed using the two approaches mentioned in chapter 4.
\end{itemize}
\begin{itemize}
    \item Comparison between the assessment of the open-source application using Prowler and ScoutSuite.
\end{itemize}
\par \textbf{Chapter 6 (Conclusion and Future Work)} concludes this thesis by summarizing the work done, results obtained, open topics, and future work.

