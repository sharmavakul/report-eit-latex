%! Author = vsharma
%! Date = 25.09.2022
% !TeX spellcheck = en_EN

\chapter{Evaluation}

\par In this chapter, we evaluate the efficiency of Prowler against security vulnerabilities based on the approaches from the previous chapter. The result of the evaluation of each approach is discussed here.

\par In the end, a comparison between the assessment of security vulnerabilities performed using Prowler and
ScoutSuite is also highlighted which showcases the efficiency of each tool individually.

\section{Using Test Driven Approach}
\begin{figure}
    \centering
    \includegraphics[width=\textwidth]{assessmentgraph.png}
    \caption{Prowler Assessment Graph}
    \label{fig:prowlerefficiency}
\end{figure}

\par The test-driven approach requires developing the test cases for the possible assessable scenario of a check in Prowler that performs the security assessment of an identified security vulnerability in the AWS service.
The individual test cases are developed, and their results are noted.

\par The checks in Prowler corresponding to the identified security vulnerabilities are identified manually.
The functionality of the check is understood and verified if the check can perform the assessment of the identified security vulnerability.
The verification process includes developing test cases for all assessable scenarios and evaluating the result of the assessment.
Only if all the scenarios are successfully verified the security vulnerability is marked to be assessed using Prowler.
Once all the test scenarios for the different checks in Prowler are developed and verified, the efficiency of Prowler is calculated against the identified security vulnerabilities for individual AWS services.
The efficiency calculation is done programmatically.

\begin{longtable}{|p{10cm}|p{2.4cm}|p{2cm}|}
    \hline
    \textbf{Security Vulnerabilities} & \textbf{Service} & \textbf{Prowler Check}\\
    \hline
    Avoid the use of the root account & IAM & Check11 \\
    \hline
    MFA is enabled for all IAM users & IAM & Check12 \\
    \hline
    Credentials unused for 90 days or greater are disabled & IAM & Check13 \\
    \hline
    Access keys are rotated every 90 days or less & IAM & Check14 \\
    \hline
    IAM password policy requires at least one uppercase letter & IAM & Check15 \\
    \hline
    IAM password policy requires at least one lowercase letter & IAM & Check16 \\
    \hline
    IAM password policy requires at least one symbol & IAM & Check17\\
    \hline
    IAM password policy requires at least one number & IAM & Check18\\
    \hline
    IAM password policy requires a minimum length of 14 or greater & IAM & Check19\\
    \hline
    IAM password policy prevents password reuse & IAM & Check110\\
    \hline
    IAM password policy expires passwords within 90 days or less & IAM & Check111\\
    \hline
    Password expiration requires an administrator reset & IAM & \\
    \hline
    Allow users to change their own password & IAM& \\
    \hline
    Insider Threat & IAM & Extra774\\
    \hline
    Access key for the root account & IAM & Check112 \\
    \hline
    Instances created from Malicious AMI & EC2 & Check76\\
    \hline
    User data public exposure & EC2 & Extra741\\
    \hline
    Server-Side Request Forgery & EC2 & Extra786\\
    \hline
    Denial of Wallet & EC2, Lambda &\\
    \hline
    Public exposure of S3 buckets & S3 & Check73\\
    \hline
    Unencrypted S3 buckets & S3 & Extra764\\
    \hline
    GhostWriter & S3 & Extra771 \\
    \hline
    Publicly accessible RDS instances & RDS & Check78\\
    \hline
    Unencrypted RDS Instance & RDS & Extra735\\
    \hline
    Resources running in an AWS classic resource & RDS & \\
    \hline
    Default data retention & RDS & Extra739\\
    \hline
    Lambda functions have a public resource-based policy & Lambda & Extra798 \\
    \hline
    Publicly accessible AWS account & Lambda & Extra7145 \\
    \hline
    Public lambda function URL & Lambda & Extra7179 \\
    \hline
    Public lambda function URL Cors & Lambda & Extra7180 \\
    \hline
    Insecure Management of Secrets & Lambda & Extra760\\
    \hline
    Poisoning the Well & Lambda & \\
    \hline
    \caption{Prowler Checks and mapped Security Vulnerabilities}
    \label{tab:securityvulnerabilitiescheckin prowler}
\end{longtable}

\par The process for calculating the efficiency of Prowler begins by determining the count of the security vulnerabilities that are identified for an AWS service.
The table \ref{tab:classificationofsecurityvulnerabilities} shows all the security vulnerabilities that are identified in the five AWS services.
Additionally, the table also highlights the different checks in Prowler that are identified and verified by developing the test cases that assess these identified security vulnerabilities.
The count of number of security vulnerabilities that are identified for the AWS service is determined.
Also, the number of security vulnerabilities that are assessed using Prowler for that AWS service is determined.
Based on these results the efficiency of Prowler is calculated in assessing an AWS service.
For example, looking at the table \ref{tab:classificationofsecurityvulnerabilities} and the graph \ref{fig:prowlerefficiency} considering AWS RDS, the number of identified security vulnerabilities is 4, and the number of security vulnerabilities that are assessed using Prowler is 3.
This process is performed for the 5 AWS services that are chosen for this research work.
In the end, it is possible to determine how efficient Prowler is in assessing the security vulnerabilities of each AWS service.


\par The graph \ref{fig:prowlerefficiency} is drawn between the AWS services considered for this thesis work and the number of security vulnerabilities.
It highlights the total number of security vulnerabilities identified for an AWS service and the number of security vulnerabilities that are assessed using Prowler for the AWS service.

\par Looking at the graph for IAM, it can be concluded that out of the 15 identified security vulnerabilities, Prowler performs the assessment of 13 security vulnerabilities.
The 2 security vulnerabilities that cannot be assessed by Prowler are \textit{Password expiration requires administrator reset} and \textit{Allow users to change their own password}.
These security vulnerabilities are caused due to excessive privilege.
The Password expiration requires an administrator reset feature prevents the user from updating their password after the account password has expired and thus requires administrative access.
This causes an issue on the user’s productivity and increases the workload on the administrator but ensures account safety.
If the account user is given administrative permission, this avoids the impact on the user’s productivity, but at the same time increases the risk that the user can cause.
This user can reset passwords for another user, revoke or grant unrestricted access to resources, files or directories and can even delete other user resources or accounts.
The allows users to change their own password feature enables all the users to change their own account password thus ensuring only the user of the account should be able to change the password for his account.
If the user is granted an administrative role, the user would not just be able to change the password for his own account but would be able to manage other user accounts as well.
Assigning excessive privilege to a user can lead to 
deadly sin such as siphoning important information to the
competitors, data destruction, etc \cite{87}.

\par Like IAM, looking at the graph for EC2 it can be determined that out of the 4 identified security vulnerabilities, 3 of those security vulnerabilities are assessed using Prowler.
 The unassessed security vulnerability using Prowler is the Denial of wallet.
 Most of the time data loss incidents make the news, but the denial of wallet is one of the common ways that can be found out about a compromise through the AWS bill.
 The denial of wallet security vulnerability causes 
disruption of the target system or flooding the system 
with traffic as a form of threat, ransom, or revenge thus
depriving the system usage or leading to downtime 
resulting in loss of time and money, and reputation \cite{86}.

\par Prowler shows 100 \% efficiency in assessing the security vulnerabilities that are identified in S3 and thus Prowler is highly efficient in performing the security best practices assessments, audits, and incident response for S3.

\par Similar to IAM and EC2, out of 4 identified security
vulnerabilities prowler performs the assessment of 3 security vulnerabilities.
Prowler does not perform the assessment of security vulnerability that can occur due to RDS instances running on AWS classic resources.
The EC2-Classic platform is retired by Amazon and disabled on all accounts.
The EC2-Classic platform is replaced the EC2-VPC \cite{88}.

\par As seen from the graph \ref{fig:prowlerefficiency}, out of the 6 security vulnerabilities identified in AWS lambda, Prowler can perform the assessment of 5 security vulnerabilities.
The security vulnerability that is not assessed using Prowler is ‘Poisoning of well’.
Poisoning of well can be highly hazardous in numerous ways.
The first scenario could be where the malicious users inject fake training data with the aim of corrupting the learned model and can spread, affecting the products that draw from it.
The second scenario where poisoning of well could occur would be where an Integrated Development Environment (IDE) plugin hosting the cloud server is controlled by an attacker.
A code backdoor is introduced as the security vulnerability by the attacker into the plugin \cite{89}.
Another scenario could be the poisoning attack on the Machine Learning dataset.
Here, in order to control the prediction behavior of the model the attacker manipulates the training dataset \cite{90}.


\section{Using Open-Source Application}

\begin{figure}
    \centering
    \includegraphics[width=\textwidth]{prowlervsscoutsuite.png}
    \caption{Security Vulnerabilities Assessment Graph}
    \label{fig:prowlervsscoutsuite}
\end{figure}

\par The employee management web application \cite{69} leverages different AWS services namely IAM, EC2, S3, RDS, and Lambda for its deployment on AWS cloud infrastructure as shown in the figure \ref{fig:infrastructure_architecture}.
Once the application is deployed on the AWS cloud, the application functionality is verified.
The application provides different functions such as adding, updating and deleting the organization’s employee information.
As the application deployment takes place, there could be a chance of introduction of security vulnerability due to misconfiguration or security flaws within the application.
In order to deal with these issues, the AWS account must be assessed using the assessment tools against any security vulnerability that might have been introduced while deploying the application on the AWS infrastructure.

\par To assess the AWS account against any security vulnerabilities two security assessment tools namely Prowler and ScoutSuite were introduced in chapter 4.
To begin the assessment of the AWS account using Prowler the command \textit{./prowler} is executed, on the other hand, the assessment of security vulnerabilities using ScoutSuite is performed by executing the command \textit{scout aws}.

\par Once the assessment of the AWS account finishes, the assessment report is generated.
Prowler supports the generation of assessment reports in multiple formats such as text, CSV, JSON, JSON-ASFF, JUnit-XML, and HTML.
The generated assessment report provides a detailed 
description of the security vulnerability assessed, check
in Prowler that assesses the security vulnerability, the result of the assessment, region, associated AWS service, etc \cite{75}.
Similarly, when the assessment of the AWS account performed using ScoutSuite finishes, the assessment report is generated in PDF format.
The assessment report provides a descriptive view of the 
different AWS services, the resources leveraged by each 
service, the rules available for each service, etc \cite{76}.


\par \par Looking at the graph \ref{fig:prowlervsscoutsuite}, it can be inferred that a total of 32 security vulnerabilities are identified in five AWS services based on literatures, the Open Web Application Security Project (OWASP) vulnerability list \cite{43}, Cybersecurity \& Infrastructure Security Agency (CISA) \cite{42}, International Data Corporation (IDC) \cite{41} as shown in the table \ref{tab:securityvulnerabilitiesandresources}.
After the assessment of the AWS account finishes, a total of 27 of the identified security vulnerabilities are assessed using Prowler.
There are 5 security vulnerabilities that could not be assessed using Prowler as Prowler does not provide checks to assess those 5 security vulnerabilities.
Similarly, out of the 32 identified security vulnerabilities, ScoutSuite can perform the security assessment of 21 security vulnerabilities.
There are 11 security vulnerabilities that are not assessed using the rules provided by ScoutSuite.

\begin{longtable}{|p{10cm}|p{2.4cm}|}
    \hline
    \textbf{Security Vulnerabilities} & \textbf{Identification Resource}\\
    \hline
    Insider threat & Literature \cite{91} \\
    \hline
    Misconfiguration: & OWASP, CISA, IDC  \\
    \hline
    Instances created from Malicious AMI & Literature \cite{48} \\
    \hline
    User data public exposure & OWASP \\
    \hline
    Server-Side Request Forgery & OWASP \\
    \hline
    Denial of Wallet & CISA \\
    \hline
    Public exposure of S3 buckets & OWASP, Literature \cite{92}\\
    \hline
    Unencrypted S3 buckets & Literature \cite{93}\\
    \hline
    GhostWriter & CISA \\
    \hline
    Public RDS Database instance & OWASP\\
    \hline
    Resources running in AWS classic resources & Literature \cite{94}\\
    \hline
    Default data retention & Literature \cite{95} \\
    \hline
    Data Event Injection & OWASP \\
    \hline
    Insecure Management of Secrets & CISA\\
    \hline
    Poisoning the Well & CISA \\
    \hline
    \caption{Identified Security vulnerabilities}
    \label{tab:securityvulnerabilitiesandresources}
\end{longtable}

\par Based on the assessment of security vulnerabilities performed using the two security tools, it is possible to calculate the efficiency of the tools.
The formula to calculate efficiency of the two security assessment tools mathematically is the ratio of output to input expressed as a percentage \cite{88}.
\[ efficiency = (output/input) * 100 \]

Using this formula, the overall percentage efficiency of Prowler is calculated to be 84 \% and the overall efficiency of ScoutSuite is calculated to be 65 \%.

\par Prowler boasts a number of checks that other tools miss, has thorough and considered documentation, and is a reliavle and lightweight piece of software.

\section{Comparision between Prowler and ScoutSuite}

\par Cloud service providers make tools available to secure the cloud systems, but it is ultimately the user’s responsibility to use them.
The control needed to define and implement the cloud infrastructure differs greatly from the controls used in on-premises environments.
Simply transforming the hardware servers to AWS EC2 instances won't make them secure by default.
This is because, while AWS is responsible for the security of the cloud, users are responsible for the security in the cloud.
Due to this, many companies suffer from misconfigured and poorly architected cloud infrastructure, leading to embarrassing data leaks \cite{74}.

\par There are a limited number of tools provided by cloud service providers.
For instance, AWS Security Hub only supports Center for Internet Security (CIS) and Payment Card Industry Data
Security Standard (PCI DSS) benchmarks.
Even that is limited since it cannot produce results on "monitoring" related CIS benchmarks (Section 3).
This is where third-party solutions come in handy \cite{74}.

\par This section shows the assessment of security vulnerability performed using two open-source cloud security assessment tools namely Prowler and ScoutSuite which can help strengthen the cloud security posture without breaking the bank.
In the end, a comparison between the assessment performed using Prowler and ScoutSuite against the different security vulnerabilities identified earlier is highlighted in the table \ref{tab:comparisionresultprowlervsscoutsuite}.

\par This section shows the comparison of the assessment of security vulnerability performed using two open-source cloud security assessment tools namely Prowler and ScoutSuite.
These tools help strengthen the cloud security posture without breaking the bank.
The table \ref{tab:comparisionresultprowlervsscoutsuite} highlights the result of the assessment obtained using the two tools.

\par Assessmement Process:


\par Prowler: Prowler users AWS CLI. Before running Prowler, the AWS CLI must be properly configured with a valid Access Key and Region or declare AWS variables properly.
This can be done by running the command \textit{aws configure}.
The credentials configured in the AWS CLI must be associated with a user.
In order to run all the Prowler checks, managed policies \textit{SecurityAudit} and \textit{ViewOnlyAccess} must be added to the user \cite{75}.

\begin{table}[h!]
    \begin{center}
        \caption{Prowler Extra78}
        \label{tab:prowlerextra}
        \begin{tabular}{|p{1.4cm}|p{1.7cm}|p{1.5cm}|p{4.0cm}|p{5.0cm}|}
            \hline
            \textbf{Result} & \textbf{Severity} & \textbf{CheckID} & \textbf{Check Title} & \textbf{Check Output}\\
            \hline
            Pass & Critical & 7.8 & [extra78] Ensure there are no Public Accessible RDS instances &
            No Publicly Accessible RDS instances found\\
            \hline
        \end{tabular}
    \end{center}
\end{table}

\par The assessment of security vulnerability using Prowler is performed by executing the command \textit{./prowler}.
When the command is run, Prowler authenticates the configured environment variable.
Upon successful authentication, the checks are run over all the AWS regions.
To limit the assessment to a custom profile and region, the command \textit{./prowler -p custom-profile -r us-east-1} is used.
Prowler also enables assessing a single security vulnerability by running the command \textit{./prowler -c check78}.
The check78 is a check id, this check ensures there are no Public Accessible RDS instances (verifies publicly accessible RDS instances).
Table \ref{tab:prowlerextra} shows the execution result of the assessment performed using Prowler \cite{75}.

\par If the user wants to save the assessment report for later analysis, Prowler enables saving the result in different formats such as CSV, JSON, Html, etc.
This is achieved by using the command \textit{./prowler -M csv}\cite{75}.


\par Scout Suite works on machines used to make AWS API calls, such as AWS CLI, EC2 instances, or any other tool based on AWS official SDKs. Similar to Prowler, before starting the assessment AWS CLI must be configured.
AWS CLI can be configured by running the \textit{aws configure} command.
The user credentials configured must be assigned the ReadOnlyAccess and SecurityAudit AWS Managed Policies \cite{76}.

\begin{table}[h!]
    \begin{center}
        \caption{ScoutSuite Publicly accessible RDS instances}
        \label{tab:scoutsuiterule}
        \begin{tabular}{|p{1.4cm}|p{1.7cm}|p{5.0cm}|p{6.0cm}|}
            \hline
            \textbf{Result} & \textbf{Severity} & \textbf{Rule} & \textbf{Rule Title}\\
            \hline
            Good & Danger & rds-instance-publicly-accessible & RDS Instance publicly accessible \\
            \hline
        \end{tabular}
    \end{center}
\end{table}

Once the AWS CLI environment is configured and appropriate credentials are set up, the assessment can be started with ScoutSuite.
If the user wishes to use Scoutsuite against a specific AWS IAM role, the command \textit{scout aws --profile my-aws-cli-profile} is executed.
The assessment is started by executing the command \textit{scout aws}.
ScoutSuite gathers data from APIs, performs the authenticity of the configured environment variable, and pulls info on the cloud services and various resources.
Once Scout Suite finishes auditing the environment, an HTML report will be generated \cite{77}.

Unlike checks in Prowler, ScoutSuite has rules.
To verify publicly accessible RDS instances, ScoutSuite uses the rule \textit{rds-instance-publicly-accessible}.
The result is demonstrated in table \ref{tab:scoutsuiterule} \cite{77}.






\begin{longtable}{|p{10cm}|p{2.2cm}|p{2.2cm}|}
    \hline
    \textbf{Security Vulnerabilities} & \textbf{Prowler} & \textbf{ScoutSuite}\\
    \hline
    Avoid the use of the root account & {{\color{green}\checkmark}} & {{\color{green}\checkmark}} \\
    \hline
    MFA is enabled for all IAM users & {{\color{green}\checkmark}} & {{\color{green}\checkmark}} \\
    \hline
    Credentials unused for 90 days or greater are disabled & {{\color{green}\checkmark}} & {{\color{green}\checkmark}} \\
    \hline
    Access keys are rotated every 90 days or less & {{\color{green}\checkmark}} & {{\color{green}\checkmark}} \\
    \hline
    IAM password policy requires at least one uppercase letter & {{\color{green}\checkmark}} & {{\color{green}\checkmark}} \\
    \hline
    IAM password policy requires at least one lowercase letter & {{\color{green}\checkmark}} & {{\color{green}\checkmark}} \\
    \hline
    IAM password policy requires at least one symbol & {{\color{green}\checkmark}} & {{\color{green}\checkmark}} \\
    \hline
    IAM password policy requires at least one number & {{\color{green}\checkmark}} & {{\color{green}\checkmark}} \\
    \hline
    IAM password policy requires a minimum length of 14 or greater & {{\color{green}\checkmark}} & {{\color{green}\checkmark}} \\
    \hline
    IAM password policy prevents password reuse & {{\color{green}\checkmark}} & {{\color{green}\checkmark}} \\
    \hline
    IAM password policy expires passwords within 90 days or less & {{\color{green}\checkmark}} & {{\color{green}\checkmark}} \\
    \hline
    Password expiration requires an administrator reset &  & \\
    \hline
    Allow users to change their own password &  & \\
    \hline
    Insider Threat & {{\color{green}\checkmark}} & {{\color{green}\checkmark}} \\
    \hline
    Access key for the root account & {{\color{green}\checkmark}} & {{\color{green}\checkmark}} \\
    \hline
    Instances created from Malicious AMI & {{\color{green}\checkmark}} & {{\color{green}\checkmark}} \\
    \hline
    User data public exposure & {{\color{green}\checkmark}} & {{\color{green}\checkmark}} \\
    \hline
    Server-Side Request Forgery & {{\color{green}\checkmark}} &  \\
    \hline
    Denial of Wallet &  & \\
    \hline
    Public exposure of S3 buckets & {{\color{green}\checkmark}} & {{\color{green}\checkmark}} \\
    \hline
    Unencrypted S3 buckets & {{\color{green}\checkmark}} & {{\color{green}\checkmark}} \\
    \hline
    Write access on S3 buckets (Ghostwriter) & {{\color{green}\checkmark}} & {{\color{green}\checkmark}} \\
    \hline
    Publicly accessible RDS instances & {{\color{green}\checkmark}} & {{\color{green}\checkmark}} \\
    \hline
    Unencrypted RDS Instance & {{\color{green}\checkmark}} & {{\color{green}\checkmark}} \\
    \hline
    Resources running in an AWS classic resource &  &  \\
    \hline
    Default data retention & {{\color{green}\checkmark}} & {{\color{green}\checkmark}} \\
    \hline
    Lambda functions have a public resource-based policy & {{\color{green}\checkmark}} &  \\
    \hline
    Publicly accessible AWS account & {{\color{green}\checkmark}} &  \\
    \hline
    Public lambda function URL & {{\color{green}\checkmark}} &  \\
    \hline
    Public lambda function URL Cors & {{\color{green}\checkmark}} &  \\
    \hline
    Insecure Management of Secrets & {{\color{green}\checkmark}} &  \\
    \hline
    Poisoning the Well &  & \\
    \hline
    \caption{Prowler vs ScoutSuite security vulnerability assessment }
    \label{tab:comparisionresultprowlervsscoutsuite}
\end{longtable}